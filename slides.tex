\documentclass[EU2]{beamer}
% need fontspec for Unicode characters, but don't need font
\usepackage{fontspec}
\usepackage[ngerman]{babel}
\usepackage{csquotes}
\mode<presentation>{\usetheme{Copenhagen}}
\title{Hollerith und die D11}
\author{Simon~Becker, Marek~Kubica}
\date{22.~Mai~2013}
\institute{Technische Universität München}

\hypersetup{pdfencoding=auto}
%\fontspec{Latin Modern Roman}

\beamertemplatenavigationsymbolsempty

% seitenzahlen einfügen
% http://texblog.net/latex-archive/plaintex/beamer-footline-frame-number/
\newcommand*\oldmacro{}%
\let\oldmacro\insertshorttitle%
\renewcommand*\insertshorttitle{%
  \oldmacro\hfill%
  \insertframenumber\,/\,\inserttotalframenumber}

\begin{document}

\frame{\titlepage}

\logo{\includegraphics[height=1cm]{in_tum}}

\frame{\tableofcontents}

\section{Herman Hollerith}

\subsection{Anfänge}

\begin{frame}{Lebenslauf}
  \begin{columns}
    \column{0.6\textwidth}
    \begin{itemize}
      \item Geboren 29. Februar 1860 in Buffalo, NY
      \item Kind Deutscher Einwanderer
      \item Entwickler von Tabelliermaschinen
      \item Pionier des Einsatzes von Elektrizität
      \item Erfinder (unter anderem Zugbrems-Systeme)
      \item Heirat mit Lu Talcott
      \item Gestorben 17. November 1929
    \end{itemize}
    \column{0.4\textwidth}
      \includegraphics[height=0.8\textheight]{hollerith}\\
      \tiny{Geoffrey D. Austrian, Herman Hollerith — Forgotten Giant of Information Processing}
  \end{columns}
\end{frame}

\subsection{Volkszählungen}

\begin{frame}{1890, USA}
  \begin{itemize}
    \item Vorher in einem Büro der Volkszählung gearbeitet
    \item Ausschreibung für Tabelliersysteme
      \begin{enumerate}
	\item 72~Stunden (Daten) + 5 (Tabellierung) Hollerith
	\item 110 + 44~Stunden Pidgin
	\item 144 + 55~Stunden Hunt
      \end{enumerate}
    \item Innerhalb 6~Wochen verarbeitet
    \item Eine Arbeitskraft verarbeitet 50.000 Personen pro Tag
    \item Gründung von \enquote{The Tabulating Machine Company} in Georgetown
  \end{itemize}
\end{frame}

\begin{frame}[plain]{Bahnticket}
  \begin{center}
    \includegraphics[width=\textwidth]{phototicket}\\
    \tiny{\url{http://discussion.cprr.net/2007/01/punch-photograph-railroad-ticket.html}}
  \end{center}
\end{frame}

\begin{frame}[plain]{Pantograph Punch}
  \begin{center}
    \includegraphics[height=0.9\textheight]{pantograph}\\
    \tiny{\url{https://www.census.gov/history/www/innovations/technology/the\_hollerith\_tabulator.html}}
  \end{center}
\end{frame}

\begin{frame}[plain]{1890er Tabelliermaschine}
  \begin{center}
    \includegraphics[height=0.9\textheight]{manuell}\\
    \tiny{\url{https://secure.flickr.com/photos/44124384537@N01/411109339}}
  \end{center}
\end{frame}

\begin{frame}{1900, USA}
  \begin{itemize}
    \item TMC wird zur Aktiengesellschaft
    \item Wichtig: Farmstatistiken $\Rightarrow$ Erweiterung der
      Inkrementiermaschinen zu Addiermaschinen
    \item \enquote{Key Punch} Lochstanzmaschine
    \item ab 1901 mit automatischer Sortiermaschine
  \end{itemize}
\end{frame}

\begin{frame}[plain]{Key Punch}
  \begin{center}
    \includegraphics[width=1.09\textwidth]{keypunch}\\
    {\tiny \url{http://www.officemuseum.com/data_processing_machines.htm}}\\
    (Späteres Modell)\\
  \end{center}
\end{frame}

\begin{frame}{Zählungen in anderen Ländern}
  \begin{block}{England}
    \begin{itemize}
      \item Sabotage durch Arbeiter
    \end{itemize}
  \end{block}
  \begin{block}{Österreich}
    \begin{itemize}
      \item Nachbau der Maschinen
    \end{itemize}
  \end{block}
  \begin{block}{Russland}
    \begin{itemize}
      \item Zähe Verhandlungen mit Österreichern und Hollerith
      \item Letztendlich auf Hollerith gesetzt
    \end{itemize}
  \end{block}
  Weitere Länder wie Norwegen, Kanada setzen das System ein.
\end{frame}

\subsection{Kommerzieller Einsatz}

\begin{frame}{Konflikte}
  \begin{block}{Westinghouse}
    Konflikt wegen Bremssystemen für Eisenbahnbremsen, statt hydraulischer
    Signalisierung nutzt Hollerith elektrische, Westinghouse setzt sich jedoch
    zunächst durch.
  \end{block}
  \begin{block}{Österreich}
    Die Österreicher versuchen Holleriths Erfindung selbst zu vermarkten,
    deklarieren Patente für ungültig.
  \end{block}
  \begin{block}{North}
    Organisiert neue Volkszählung, versucht Holleriths Quasi-Monopol zu brechen
    und entwickelt mit Staatsmitteln Tabelliermaschinen (mit Hilfe von James Powers).
  \end{block}
\end{frame}

\begin{frame}{New York Central}
  \begin{itemize}
    \item Eisenbahngesellschaft
    \item Nutzen Holleriths Maschinen schon relativ früh für Buchhaltung
    \item Testballon für komerzielle Vermarktung
  \end{itemize}
  % TODO bild einfügen
\end{frame}

\begin{frame}{Verkauf an C-T-R}
  \begin{itemize}
    \item Charles R. Flint überzeugt Hollerith zu einer Fusion
      \begin{enumerate}
        \item Computing Scale Company
        \item Tabulating Machine Company
	\item International Time Recording Company
	\item Bundy Manufacturing Company
      \end{enumerate}
    \item Neue Firma, Computing Tabulating Recording Company
    \item Thomas J. Watson leitet CTR
    \item CTR wird 1924 zu International Business Machines Corporation, IBM
    \item Hollerith scheidet aus dem Vorstand aus, zeitweise consulting
  \end{itemize}
\end{frame}

\begin{frame}{Geschäftsmodelle}
  \begin{itemize}
    \item Anfangs: Vermietung der Maschinen
    \item Maschinen kostenlos, Lochkarten werden abgerechnet
    \item \enquote{Cloud-Computing}
  \end{itemize}
\end{frame}

\section{DEHOMAG D11}

\begin{frame}[plain]{Dehomag D11 = IBM 450}
  \begin{center}
    \includegraphics[height=0.95\textheight]{d11}\\
    \tiny{Eigene Aufnahme}
  \end{center}
\end{frame}

\subsection{Bedeutung}

\begin{frame}{D11}
  \begin{itemize}
    \item DEHOMAG, Deutsche Hollerith-Maschinen Gesellschaft mbH
    \item D11 1935 - 1960 in Deutschland gebaut
    \item Erste in großem Maßstab in Deutschland gebaute Maschine
    \item Fabrik in Berlin errichtet
    \item Wir haben eine \emph{funktionsfähige} D11 im Deutschen Museum
    \item \emph{Video}
  \end{itemize}
\end{frame}
% Source: Punched-Card Systems and the Early Information Explosion

\subsection{Funktionsweise}

\begin{frame}{Unterschiede zu Holleriths Maschinen}
  \begin{itemize}
    \item Bedruckte Lochkarten
    \item Drucker für Ergebnisse
    \item D11 kann Subtrahieren, Multiplizieren und Dividieren
    \item Verarbeitungsschritte können \emph{programmiert} werden
  \end{itemize}
\end{frame}

\section{Die Zeit nach der D11}

\subsection{Weiterentwicklung der D11}

\begin{frame}{Kritikpunkte an der D11}
    \begin{itemize}
      \item Unzureichender Zeichenvorrat
      \item Zu langsame Druckgeschwindigkeit
      \item Zu geringe arithmetische Leistung
    \end{itemize}
\end{frame}

\begin{frame}{Tabelliermaschine IBM 407}
   \begin{columns}
    \column{0.6\textwidth}
    \begin{itemize}
      \item Erscheinungsjahr: 1949
      \item Markt: USA
      \item Neuerung: Typenrad
      \item Vorteile des neuartigen Druckwerk
      	\begin{itemize}
          \item Erhöhung der Druckgeschwindigkeit
          \item Qualitativ gutes Druckbild
       \end{itemize}
    \end{itemize}
   \column{0.4\textwidth}
      \includegraphics[height=0.4\textheight]{typenrad}\\
      \tiny{G. Sandner,H. Spengler(2006):Die Entwicklung von Hollerith Lochkartenmaschinen zu IBM Enterprise-Servern}
  \end{columns}
\end{frame}

\begin{frame}[plain]{Tabelliermaschine IBM 407}
\begin{center}
    \includegraphics[height=0.95\textheight]{IBM407}\\
    \tiny{\url{http://www.piercefuller.com/library/p8041005a.html?id=p8041005a}}
  \end{center}
\end{frame}

\begin{frame}{Tabelliermaschine IBM 421}
    \begin{itemize}
      \item Erscheinungsjahr: Anfang 1950er
      \item Markt: Europa, Asien, Afrika
      \item Konnte an gewünschte Arbeiten angepasst werden
      \item Schnellere Druckgeschwindigkeit
    \end{itemize}
\end{frame}

\begin{frame}[plain]{Tabelliermaschine IBM 421}
\begin{center}
    \includegraphics[height=0.95\textheight]{IBM421}\\
    \tiny{G. Sandner,H. Spengler(2006):Die Entwicklung von Hollerith Lochkartenmaschinen zu IBM Enterprise-Servern}
  \end{center}
\end{frame}

\subsection{Geschichte der Dehomag}
\begin{frame}{Geschichte der Dehomag}
    \begin{itemize}
      \item 1910 von Willy Heidinger gegründet
      \item Produktion von Lochern, Sortierern und Tabelliermaschinen
      \item Kunden: u.a. statistische Ämter, Industrie und in der NS-Zeit die Wehrmacht und Schutzstaffel
      \item 1922 90\%ige Übernahme durch IBM
      \item Unternehmenswachstum mit D11
      \item 1949 Umbenennung in IBM Deutschland
    \end{itemize}
\end{frame}

\subsection{Ende der Lochkarte als Speichermedium}
\begin{frame}{Ende der Lochkarte als Speichermedium}
  \begin{itemize}
    \item Magnetische Speicherung löst in den 1960er Jahren die Lochkarte als Medium der Massenspeicherung ab
    \item Vorteil magnetischer Speicherung:
        \begin{itemize}
          \item Wiederverwendbarkeit
          \item Geringere Kosten
	  \item Platzsparender
       \end{itemize}
 \end{itemize}
\end{frame}

\subsection{Lochkartentechnik heute}
\begin{frame}{Lochkartentechnik heute}
  \begin{itemize}
    \item US-Wahl
    \item Firma \enquote{Sparkler Filters}
    \item IBM Projekt \enquote{Millipede}
        \begin{itemize}
          \item Speichertechnik im Nanometerbereich
          \item Grundprinzip der Lochkartentechnik
	  \item Winzige Messnadeln schreiben, lesen oder löschen Bits in einem Polymerfilm
       \end{itemize}
  \end{itemize}
\end{frame}

% Wir sind mit den Sections fertig, leere Section für Schlussslide
\section*{}

\begin{frame}{Fragen}
  \begin{center}
    \Large{Danke fürs Zuhören!}
  \end{center}
  \begin{center}
    $\vdots$
  \end{center}
  \begin{center}
    \Large{Fragen?}
  \end{center}
\end{frame}

\end{document}

% Volkszählung 1890
% Anfangs Job im Zensusbüro
% Idee zu einer Tabelliermaschine um die mechanische Arbeit zu verrichten
% Lanston-Kooperation zu Addiermaschinen
% Vortragender am MIT (kurzzeitig), 1882
% Idee die Löcher in Lochkarten für Numerische Werte stehen zu lassen
% Idee der Lochkarten: von punch photographs auf eisenbahnticket
% http://discussion.cprr.net/2007/01/punch-photograph-railroad-ticket.html
% Unterschied zu Jaquard: Elektrizität
% 1884 erstes patent auf Tabelliermaschine für Volkszählungen
%
% Experimente mit elektrischen Bremssystemen für Züge 1886, 1887, luftbremsen mit elektrosignalisierung statt pneumatik
% erst 1930 wiedererfunden
%
% Karten statt einem langen Band weil einfacher zu lagern, verarbeiten, erstellen
% Robert P. Porter als superindentant für 1890 US Volkszählung
% Ausschreibung für Tabelliermaschinen für 1890
% Konkurenten: Charles F. Pidgin, Mr Hunt
% Neue Punch: pantograph punch
% Testdaten in 72 stunden aufgenommen vs 110 pidgin und 144 hunt
% tabelliermaschine: 5h statt 44h Pidgin und 55 hunt
%
% Ernstfall: 1890: 50.000 Personen pro Tag gezählt durch eine Arbeitskraft
% Karten unbedruckt, daher einfacher für Maschine als für Mensch zu lesen
% Auswertung von mehreren werten gleichzeitig durch Relays
% 6 Wochen durchgelaufen
%
% Geschäftsmodell: verleih statt verkauf, weil die Maschinen professionelle Wartung erfordern
%
% Heirat im schnellverfahren mit Lu Talcott
%
% Volkszählung im AUsland: Österreich, Kanada und Norwegen
%
% 1892 Georgetown, "The Tabulating Machine Company"
% Produktion anfangs durch Western Electric (tabul, sorter) und Pratt & Whitney (punch)
% Nach der Volkszählung: alle 105 Tabul wurden zurückgegeben -> fast-pleite
%
% Erweiterung von Inkrementierern auf Addierer -> Accounting
% Feststellung: Besser Tabul konstant an komerzielle Kunden anbieten statt Zensus
%
% Volkszählung in Russland 1895 von Alexander III angeordnet
% Negotiationen mit Hollerith und Österreich wegen Tabulatoren
%
% Einsatz in New York Central (RR)
% Bedruckte Lochkarten mit Separierungsstreifen für Blöcke
% Teilweise wurden die Berechnungen auf die Maschinen in Georgetown ausgelagert
% Ein frühes bespiel von Remote-Computing, Cloud
%
% Tabulating Machine Company, als Aktiengesellschaft
% 1000 Aktien, Hollerith 502
%
% US Volkszählung 1900
% Ausschreibung, Hollerith 185h, Pidgin 452
% Wichtig: Farmstatistiken
% Neue Punch, key punch (einhändige schreibmaschine)
% Automatic Tabulator (415 karten pro Minute), 80k-90k pro tag
% Lange Zeit keine automatischen sortiermaschinen
% 1901 sortiermaschine mit 12 taschen
%
% 1901 Tarf-Peirce Company angekauft, die die Tabelliermaschinen bauten
% Anfragen von Retailern 1903
% Tabelliermaschinen kostenlos, alle karten von TMC gekauft und nur für eine bestimmte aufgabe verwendet
%
% Zerwerfung mit North, Zensusdirektor
% North versucht eigene tabelliermaschinen im rahmen des Zensus zu bauen
% 1905 werden Holleriths maschinen aus den Zensusbüro entfernt
%
% Hollerith hört auf mit Zensus, nur noch kommerz
% Kommerzieller Erfolg bei RR
% 1908 Ressourcenknappheit, Lieferverzug
% 1909 wieder im Lot
%
% North baut maschinen, versucht Patente zu umgehen
% Wirbt Holleriths ehemalige mitarbeiter ab
% Mr. Powers, begabter Ingenieur bei North
% Powers punch, erst geschrieben wenn Operator alles absegnet
% Keine Ausschreibung für 1910 Zensus, North muss gehen
%
% Klage gegen Zensusbüro die alten Hollerith-Maschinen umbauen zu wollen
% Gescheitert
% Powers/North-Maschinen langsamer
% Powers geht (später) zu Remington Rand
%
% 1911 zustimmung zu Fusion mit drei anderen Firmen zu CTR:
% Computing-Tabulating-Recording Company aus
% Computing Scale Company of Dayton
% Tabulatimg Machine Company
% International Time Recording Company
% Bundy Manufacturing Company
%
% Hollerith wohlhabend, baut Farm
%
% 1911 CTR fängt an Maschinen selbst zu bauen
%
% 1910 Willy Heidinger interessiert sich für Hollerith's Maschinen
% Deutsche Hollerith Maschinen Gesellschaft, Dehomag
% Lizenzzahlungen an TMC/CTR
% Watson an CTR-Spitze
% Watson lizensiert Hollerith-Patente an Powers/Remington Rand
%
% Hollerith stirbt 17. November 1929
