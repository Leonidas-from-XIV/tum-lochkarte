% Momentan eine Sammlung an Punkten
% Themenbereiche:
% Herman Hollerith und die Hollerith-Tabelliermaschienen
% Dehomag und D11
% D11 danach

% Volkszählung 1890
% Anfangs Job im Zensusbüro
% Idee zu einer Tabelliermaschiene um die mechanische Arbeit zu verrichten
% Lanston-Kooperation zu Addiermaschienen
% Vortragender am MIT (kurzzeitig), 1882
% Idee die Löcher in Lochkarten für Numerische Werte stehen zu lassen
% Idee der Lochkarten: von punch photographs auf eisenbahnticket
% http://discussion.cprr.net/2007/01/punch-photograph-railroad-ticket.html
% Unterschied zu Jaquard: Elektrizität
% 1884 erstes patent auf Tabelliermaschiene für Volkszählungen
%
% Experimente mit elektrischen Bremssystemen für Züge 1886, 1887, luftbremsen mit elektrosignalisierung statt pneumatik
% erst 1930 wiedererfunden
%
% Karten statt einem langen Band weil einfacher zu lagern, verarbeiten, erstellen
% Robert P. Porter als superindentant für 1890 US Volkszählung
% Ausschreibung für Tabelliermaschienen für 1890
% Konkurenten: Charles F. Pidgin, Mr Hunt
% Neue Punch: pantograph punch
% Testdaten in 72 stunden aufgenommen vs 110 pidgin und 144 hunt
% tabelliermaschiene: 5h statt 44h Pidgin und 55 hunt
%
% Ernstfall: 1890: 50.000 Personen pro Tag gezählt durch eine Arbeitskraft
% Karten unbedruckt, daher einfacher für Maschiene als für Mensch zu lesen
% Auswertung von mehreren werten gleichzeitig durch Relays
% 6 Wochen durchgelaufen
%
% Geschäftsmodell: verleih statt verkauf, weil die Maschienen professionelle Wartung erfordern
%
% Heirat im schnellverfahren mit Lu Talcott
%
% Volkszählung im AUsland: Österreich, Kanada und Norwegen
%
% 1892 Georgetown, "The Tabulating Machine Company"
% Produktion anfangs durch Western Electric (tabul, sorter) und Pratt & Whitney (punch)
% Nach der Volkszählung: alle 105 Tabul wurden zurückgegeben -> fast-pleite
%
% Erweiterung von Inkrementierern auf Addierer -> Accounting
% Feststellung: Besser Tabul konstant an komerzielle Kunden anbieten statt Zensus
%
% Volkszählung in Russland 1895 von Alexander III angeordnet
% Negotiationen mit Hollerith und Österreich wegen Tabulatoren
%
% Einsatz in New York Central (RR)
% Bedruckte Lochkarten mit Separierungsstreifen für Blöcke
% Teilweise wurden die Berechnungen auf die Maschienen in Georgetown ausgelagert
% Ein frühes bespiel von Remote-Computing, Cloud
%
% Tabulating Machine Company, als Aktiengesellschaft
% 1000 Aktien, Hollerith 502
%
% US Volkszählung 1900
% Ausschreibung, Hollerith 185h, Pidgin 452
% Wichtig: Farmstatistiken
% Neue Punch, key punch (einhändige schreibmaschiene)
% Automatic Tabulator (415 karten pro Minute), 80k-90k pro tag
% Lange Zeit keine automatischen sortiermaschienen
% 1901 sortiermaschiene mit 12 taschen
%
% 1901 Tarf-Peirce Company angekauft, die die Tabelliermaschienen bauten
% Anfragen von Retailern 1903
% Tabelliermaschienen kostenlos, alle karten von TMC gekauft und nur für eine bestimmte aufgabe verwendet
%
% Zerwerfung mit North, Zensusdirektor
% North versucht eigene tabelliermaschienen im rahmen des Zensus zu bauen
% 1905 werden Holleriths maschienen aus den Zensusbüro entfernt
%
% Hollerith hört auf mit Zensus, nur noch kommerz
% Kommerzieller Erfolg bei RR
% 1908 Ressourcenknappheit, Lieferverzug
% 1909 wieder im Lot
%
% North baut maschienen, versucht Patente zu umgehen
% Wirbt Holleriths ehemalige mitarbeiter ab
% Mr. Powers, begabter Ingenieur bei North
% Powers punch, erst geschrieben wenn Operator alles absegnet
% Keine Ausschreibung für 1910 Zensus, North muss gehen
%
% Klage gegen Zensusbüro die alten Hollerith-Maschienen umbauen zu wollen
% Gescheitert
% Powers/North-Maschienen langsamer
% Powers geht (später) zu Remington Rand
%
% 1911 zustimmung zu Fusion mit drei anderen Firmen zu CTR:
% Computing-Tabulating-Recording Company aus
% Computing Scale Company of Dayton
% Tabulatimg Machine Company
% International Time Recording Company
% Bundy Manufacturing Company
%
% Hollerith wohlhabend, baut Farm
%
% 1911 CTR fängt an Maschienen selbst zu bauen
%
% 1910 Willy Heidinger interessiert sich für Hollerith's Maschienen
% Deutsche Hollerith Maschienen Gesellschaft, Dehomag
% Lizenzzahlungen an TMC/CTR
% Watson an CTR-Spitze
% Watson lizensiert Hollerith-Patente an Powers/Remington Rand
%
% Hollerith stirbt 17. November 1929
