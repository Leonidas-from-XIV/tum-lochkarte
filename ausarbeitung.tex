\documentclass[twocolumn]{scrartcl}
\usepackage{fontspec}
\usepackage{hyperref}

\fontspec{Latin Modern Roman}

\author{Simon Becker, Marek Kubica, Antonia Schmalstieg}
\title{Die Geschichte der Rechnerarchitektur\\
Sommersemester 2013\\
\textbf{Verarbeitung von Lochkarten: Hollerith und die D11}}

\date{22. März 2013}

\begin{document}
\maketitle

\begin{abstract}

Die Zusammenfassung ist eine Kurzfassung des Artikels. Sie sollte
auch getrennt vom restlichen Text verständlich sein
und bildet eine abgeschlossene Einheit. Insbesondere sollten keine
Verweise auf andere Abschnitte
oder das Literaturverzeichnis gemacht werden. Auf eine Einleitung
und einen Schluß kann hier verzichtet werden, um Platz zu sparen.

Ziel der Zusammenfassung ist in erster Linie, den Lesern die Wahl
zu erleichtern, welche Artikel sie lesen sollen. Folge: ist die
Zusammenfassung schlecht, schwammig oder langweilig, wird der
Artikel seltener gelesen.
\end{abstract}

\section{Einleitung}

% Label können frei gewählt werden. Später kann mittels \autoref{Einleitung}
% auf diesen Abschnitt verwiesen werden
\label{sec:einleitung}

Die Einleitung einer wissenschaftliochen Arbeit besteht üblicherweise
aus den Abschnitten
\begin{itemize}
	\item Motivation,
	\item Problem- bzw. Fragestellung und
	\item Überblick.
\end{itemize}

\bibliographystyle{plain}
\bibliography{ausarbeitung}

\end{document}
