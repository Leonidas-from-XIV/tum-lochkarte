\documentclass{scrartcl}
\usepackage{fontspec}
\usepackage{hyperref}
\usepackage{csquotes}

\fontspec{Latin Modern Roman}

\author{Simon Becker, Marek Kubica, Antonia Schmalstieg\\ TU München}
\title{Die Geschichte der Rechnerarchitektur\\
Sommersemester 2013\\
\textbf{Verarbeitung von Lochkarten: Hollerith und die D11}}

\date{22. März 2013}

\begin{document}
\maketitle

\begin{abstract}

Die Zusammenfassung ist eine Kurzfassung des Artikels. Sie sollte
auch getrennt vom restlichen Text verständlich sein
und bildet eine abgeschlossene Einheit. Insbesondere sollten keine
Verweise auf andere Abschnitte
oder das Literaturverzeichnis gemacht werden. Auf eine Einleitung
und einen Schluß kann hier verzichtet werden, um Platz zu sparen.

Ziel der Zusammenfassung ist in erster Linie, den Lesern die Wahl
zu erleichtern, welche Artikel sie lesen sollen. Folge: ist die
Zusammenfassung schlecht, schwammig oder langweilig, wird der
Artikel seltener gelesen.
\end{abstract}

\section{Einleitung}

% Label können frei gewählt werden. Später kann mittels \autoref{Einleitung}
% auf diesen Abschnitt verwiesen werden
\label{sec:einleitung}

Die Einleitung einer wissenschaftlichen Arbeit besteht üblicherweise
aus den Abschnitten
\begin{itemize}
	\item Motivation,
	\item Problem- bzw. Fragestellung und
	\item Überblick.
\end{itemize}


\section{Herman Hollerith}


\section{Unterschiedliche Lochkartenpatente}

\subsection{Tabelliermaschine}
\subsection{Lochkartensortierer}
\subsection{Lochkartenlocher}
\subsection{Lochkartenleser}

\section{Das Hollerith-Lochkartenverfahren}

\section{Von Dehomag zu IBM}

Mit der Erfindung der D11 konnte die Dehomag in den folgenden Jahren sowie
während des zweiten Weltkrieges Gewinne erwirtschaften. Allein in den ersten
acht Jahren wurden mehr als 1100 Tabelliermaschinen vom Typ D11 ausgeliefert
(Kistermann 1995, 47). Die Dehomag wurde stetig größer und 1940 war sie bereits
in Deutschland und Österreich in 59 Städten vertreten und beschäftigte mehr als
2500 Mitarbeiter (Dingwerth 2008, 144). 1960 wurde schließlich die letzte D11
gebaut und obwohl es die ersten programmierbaren Rechner gab, waren zu der Zeit
noch ungefähr 200 D11 Maschinen in Betrieb. Allerdings wurden die D11 in den
nächsten Jahren immer weniger genutzt und durch Computer ersetzt Heute steht
eine der letzten restaurierten D11 Tabelliermaschinen in München im Deutschen
Museum (Kistermann 1995, 45).

Zu den Kunden der Dehomag gehörten unter anderem statistische Ämter, die
Industrie und während des Krieges die Wehrmacht sowie die Schutzstaffel der
Nationalsozialisten. Aufgrund des Krieges konnte IBM mit seiner Tochterfirma
Dehomag nur in neutralen oder von deutschen Truppen besetzten Ländern
zusammenarbeiten. Erst nach dem Ende des Krieges 1945 wurde IBM wieder in
Deutschland geschäftlich tätig. Im Jahr 1949 wurde die Dehomag dann in
\enquote{Internationale Büro-Maschinen Gesellschaft mbH} (IBM) umbenannt,
woraus später die \enquote{IBM Deutschland GmbH} wurde (Sendler 2009, 199).
Noch heute ist IBM in ganz Deutschland vertreten und gehört zu den führenden
Unternehmen im IT- Bereich. Es ist eine deutsch-amerikanische
Erfolgsgeschichte.

\section{Weiterentwicklung der D11}

Die Dehomag Tabelliermaschine D11 verkaufte sich wie erwähnt sehr gut, aber war
nicht das letzte Gerät in der Geschichte der Lochkartentechnik. Beim Einsatz
der D11 Maschine kristallisierten sich die Kritikpunkte eines unzureichenden
Zeichenvorrates, zu langsamer Druckgeschwindigkeit und zu geringer
arithmetischer Leistung heraus (Sandner/Spengler 2006, 20). Im Jahr 1949 kam
mit der Buchungsmaschine IBM 407 eine Weiterentwicklung auf den Markt, die ein
neuartiges Druckwerk beinhaltete. Die Neuerung war ein sich nur vorwärts
drehendes, elektromagnetisch gesteuertes Rad. Dieses Rad wurde Typenrad genannt
und war der Träger des Zeichensatzes. Zusätzlich zum Alphabet und den
einstelligen Zahlen enthielt das Typenrad elf Sonderzeichen, welche nicht
chronologisch angeordnet waren (siehe \autoref{fig:typenrad}). Die Aufhängung
des Typenrades besorgte eine Schwinge, wodurch das Typenrad selbst zum Hammer
wurde. Mit dieser Technik konnte die Druckgeschwindigkeit erhöht und ein
qualitativ gutes Druckbild durch exakte Typenpositionierung gewährleistet
werden (Sandner/Spengler 2006, 20). Ein weiterer Vorteil der IBM 407 war, dass
die Operationen auf den Lochkarten mit einem einzigen Maschinenzyklus
durchgeführt werden konnten (Boyell 1957, 448). Aufgrund der Möglichkeit der
neuen Zeichenkombinationen wurden gleichzeitig neue Kartenlocher und Prüfer
entwickelt, damit die neue Technik eingesetzt werden konnte.

\begin{figure}[h]
  \centering
  \includegraphics{typenrad}
  \caption{Schematisch dargestelltes Typenrad der IBM 407 von der Seite
(Sandner/Spengler 2006, 20)}
  \label{fig:typenrad}
\end{figure}

Für die Entwicklung der IBM 407 war IBM USA verantwortlich und aufgrund der
Währungsrelation war die Monatsmiete der IBM 407 in Europa deutlich teurer als
die Miete für die Dehomag D11 (Sandner/Spengler 2006, 21). Aus diesem Grund war
die D11 in Europa verbreiteter. Anfang der 1950er Jahre kam mit der
Tabelliermaschine IBM 421 eine wichtige Weiterentwicklung für den europäischen
Markt heraus. Sie besaß eine umfassende Programmsteuerung und konnte an die
gewünschten Arbeiten ihrer Anwender angepasst werden (Weinhart 1990, 145).
Anstatt dem sich vorwärts drehendem Typenrad wurde bei der IBM 421 wieder das
Typenstangendruckprinzip genutzt, wobei es für jede Typenstange eine neuartige
Antriebssteuerung gab. Die Produktion der IBM 421 fand bei IBM Frankreich und
bei der in IBM Deutschland umbenannten Dehomag statt. \enquote{Aufgrund ihrer
Universalität war die 421 außer in Europa ein bis Nahost, Fernost, Afrika und
Südamerika gefragtes Produkt.} (Sandner/Spengler 2006, 21).

\enquote{Magnetische Speicherung ersetzt Lochkarte als Speichermedium}

Obwohl die IBM 421 in vielen Ländern zum Einsatz kam und \enquote{Anwender mit um die
10 Tabelliermaschinen in der zentralen Lochkartenabteilung} (Sandner/Spengler
2006, 22) keine Seltenheit waren, gehörte die IBM 421 zu den letzten
Weiterentwicklungen. Der Grund dafür war die Verwendung der magnetische
Speicherung von binären Daten, wodurch die Lochkarten in den 1960er Jahren als
Medium für Massenspeicherung abgelöst wurden (Gäbler/Gronau 2010, 41). Bei
dieser Speichertechnik wird beim Schreiben eine Polaritätsänderung der
Ferritteile in der  ferromagnetischen Oberfläche erzeugt. Je nachdem, ob der
Bereich Richtung Nord- bzw. Südpol zeigt, wird beim Lesen die Richtung des
Magnetfeldes als 0 oder 1 interpretiert (Gäbler/Gronau 2010, 41). Für die
magnetische Speicherung können Bänder, Karten oder Platten genutzt werden. Das
Prinzip der Magnetspeicherung findet auch bei den heutigen Festplatten
Anwendung.

Abschließend werden die Vor- und Nachteile dieser neuen Speichertechnik
genannt. Ein Vorteil von magnetischen Datenträgern gegenüber von Lochkarten ist
die Wiederverwendbarkeit. Alte Daten können gelöscht und neu überschrieben
werden. Außerdem sind die Kosten pro Gigabyte bei magnetischen Datenträgern
günstiger als bei den Lochkarten (Dee 2008, 1775). Aufgrund physikalischer und
technischer Grenzen hat der magnetisierte Bereich eine bestimmte Größe, weshalb
das Speichervolumen eines solchen Datenträger endlich ist. Ein Nachteil
magnetischer Datenträger ist somit, dass sie nicht beliebig klein sein können.

\section{Lochkarten im digitalen Zeitalter}

In diesem Abschnitt sollen Beispiele und Bereiche gezeigt werden, in denen auch
noch in dieser technisch weit vorangeschrittenen Zeit Lochkarten oder das
Prinzip der Lochkartentechnik zum Einsatz kommen. Die USA benutzt
beispielsweise vereinzelt in ihren Wahlautomaten Lochkarten, um ein schnelles
auswerten der Stimmen zu gewährleisten. So gaben bei der Präsidentschaftswahl
im Jahr 2000 die Bürger der Stadt Palm Beach im Bundesstaat Florida ihre Stimme
durch das Stanzen eines Loches in eine Lochkarte ab (Ilsemann/Simons 2000,
198). Probleme bei der Auswertung der Stimmen in Florida ließen die Lochkarten
in Verruf geraten und schließlich führte ein Urteil des Obersten Gerichtshof
der Vereinigten Staaten zum Wahlsieg von George W. Bush. Dennoch wurde bei der
bisher letzten US-Wahl 2012 \enquote{per Brief- oder Online-Wahl und - wie in
manchen Gegenden in Idaho - zum Teil auch noch mit Lochkarten} (o.V. 2012)
abgestimmt.  Bei dieser Wahl gab es keine bekannten negative Zwischenfälle. Des
Weiteren existiert in der Stadt Conroe in Texas die Firma \enquote{Sparkler
Filters}, die ihre Buchhaltungsaufgaben bis heute mit der Tabelliermaschine
Sparklers IBM 402 auf Hollerith-Lochkarten ausführt (Edwards 2012).

IBM benutzt zwar keine Tabelliermaschinen mehr, verwendet aber in seinem
Projekt \enquote{Millipede} das Grundprinzip der Lochkarten. Bei diesem Projekt
handelt es sich um eine Speichertechnik, welche im Nanometerbereich angewendet
wird.  Mit winzigen Messnadeln können Bits in einen Polymerfilm geschrieben,
gelesen oder gelöscht werden (Binnig et al. 2010, 1603). Eine Vertiefung im
Polymerfilm wird dabei als 1 und die unveränderte Oberfläche als 0
interpretiert. Der Unterschied zu herkömmlichen Lochkarten ist somit neben der
Größe die Möglichkeit Bits zu löschen und zu überschreiben. Mit dieser Technik
können bis zu ein Terabyte pro Quadratzoll (in²) gespeichert werden, wobei ein
Quadratzoll der Fläche von 6,4516cm² entspricht (Binnig et al. 2010, 1601). Die
Lese- und Schreibgeschwindigkeit kann durch Nutzung von mehreren parallel
arbeitenden Messnadeln optimiert werden.

Diese aufgezeigten Beispiele sind allerdings die Ausnahme, denn die
Lochkartentechnik ist in der heutigen Zeit veraltet und in der Computertechnik
nicht mehr von Bedeutung. Ein Zahlenbeispiel wird den technischen Fortschritt
im Bereich der Datenspeicherung belegen: Die am häufigsten verbreiteten
Lochkarten besaßen 80 Spalten und konnten somit 80 Byte speichern. Rechnet man
das auf eine 120 GB Festplatte hoch, so kommt man auf eine Anzahl von über 1,6
Milliarden Lochkarten, die die selbe Speicherkapazität wie diese Festplatte
besitzen (Roeltgen 2006, 96). Mittlerweile gibt es bereits Festplatten mit sehr
viel mehr als 120 GB Speicher.


\bibliographystyle{plain}
\bibliography{ausarbeitung}

\end{document}
